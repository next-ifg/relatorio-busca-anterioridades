
\chapter{Web of science}
\label{chap:web-science}

\section{Coffee Selective Harvesting}
\label{sec:coff-select-harv}


\ArtigoWEBOFSCIENCE{Determination of modal properties of the coffee fruit-stem
  system using high speed digital video and digital image processing}% Titulo
{Villibor, Geice Paula; Santos, Fabio Lucio; de Queiroz, Daniel Marcal;
  \textit{et al.}}% Autor(es)
{Acta scientiarum-technology}% Periódico/ Revista/ etc...
{38}% Volume
{1}% Edição
{41-48}% Paginas
{2016}% Data Publicação
{Detachment of coffee fruit is usually accomplished by means of mechanical
  impacts and vibrations applied to the plant. Modal properties of the coffee
  fruit-stem system represent important information for efficient and selective
  harvesting. This study aimed to determine the modal parameters of the coffee
  fruit-stem system, such as natural frequency and damping coefficient, using
  high speed digital videos. With image processing techniques, it was obtained
  the resulting displacement of the system subjected to an impulse. The modal
  parameters were determined by the logarithmic decrement method, considering
  the system as underdamped. The use of high-speed video and digital image
  processing techniques allowed the simple and reliable determination of modal
  parameters of the coffee fruit-stem system. Natural frequencies for the coffee
  fruit-stem system were 11.62 and 13.29 Hz; the damping coefficient was 0.0253
  and 0.029 N s m(-1), and the equivalent stiffness was 8.61 and 7.09 N m(-1)
  for red and green ripening stages, respectively. It was found overlap of
  resonance bands, between the ripening stages red and green, hindering the
  selective mechanical detachment in the first natural frequency range of the
  coffee fruit-stem system.}% Resumo


\ArtigoWEBOFSCIENCE{Spatial variability of the detachment force of coffee
  fruit}% Titulo
{Ferraz, Gabriel A. E. S.; Da Silva, Fabio M.; De Oliveira, Marcelo S.;
  \textit{et al.}}% Autor(es)
{Engenharia agricola}% Periódico/ Revista/ etc...
{34}% Volume
{6}% Edição
{1210-1223}% Paginas
{NOV-DEC 2014}% Data Publicação
{The aim of this study was to use the georeferencing and geostatistics
  techniques to evaluate the spatial variability of coffee fruit detachment
  force by semivariograms adjustments and kriging interpolation. The study was
  conducted in Tres Pontas, MG, Brazil. The detachment force of green and mature
  coffee fruit was obtained throughout a prototype dynamometer on georeferenced
  locations. The spatial dependence data was evaluated by classical and robust
  semivariogram adjustments, using ordinary least square and weighted least
  squares methods. Maximum likelihood and restricted maximum likelihood methods
  were also evaluated, but only for the classical semivariogram. Spherical,
  exponential and Gaussian models were compared for all the evaluated methods of
  semivariogram estimation. The isoline maps obtained by kriging were generated
  based on the best adjustment method and model of the semivariogram function
  obtained by the statistical validation. The studied variables showed spatial
  dependence, which were modeled by semivariograms that allowed plotisoline maps
  of the spatial distribution obtained by kriging. It was possible to identify
  the best places to start the mechanical and selective coffee fruit
  harvest.}% Resumo


\section{Coffee Harvesting Machine}
\label{sec:coff-harv-mach}

\ArtigoWEBOFSCIENCE{Geometric modeling of a coffee plant for displacements
  prediction}% Titulo
{Carvalho, Enio de Araujo; Magalhaes, Ricardo Rodrigues; Santos, Fabio
  Lucio}% Autor(es)
{Computers and electronics in agriculture}% Periódico/ Revista/ etc...
{123}% Volume
{}% Edição
{57-63}% Paginas
{APR 2016}% Data Publicação
{Coffee plants can present structural problems during semi- mechanized and
  mechanized harvesting such as excessive defoliation and breaking branches. For
  this reason, studies of mechanical response of a coffee plant can help the
  development of more optimized machines.  The main objective of this study is
  the modeling of a coffee plant for its mechanical behavior evaluation by using
  the Finite Element Analysis. The presented paper includes modeling, numerical
  simulations and experimental tests from a Coffea arabica L. plants. Firstly,
  it was necessary to create a coffee tree geometry based on pieces of a real
  tree using 3D scanning process. The coffee tree geometry together with
  experimental data provided materials properties of the wood plant which are
  used for displacements prediction via Finite Element Analysis. In order to
  validate the methodology, simulated results were compared to a real plant
  behavior under static load. Results presented consistent values from the
  three-dimensional modeling of a coffee plant which demonstrated the
  potentiality for new applications. \copyright{} 2016 Elsevier B.V. All rights
  reserved.}% Resumo


%\ArtigoWEBOFSCIENCE{}% Titulo
% {}% Autor(es)
% {}% Periódico/ Revista/ etc...
% {}% Volume
% {}% Edição
% {}% Paginas
% {}% Data Publicação
% {}% Resumo




%%% local Variables:
%%% mode: latex
%%% TeX-master: "../template-01"
%%% End:
