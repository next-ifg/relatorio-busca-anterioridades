
\chapter{INPI}
\label{chap:inpi}

\section{IPC A01D 46/06}
\label{sec:ipc-a01d-4606}

\subsection{Derriça AND Café}
\label{sec:derrica-cafe}

\PatenteINPI{Equipamento recolhedor de café, frutas e outros grãos derriçados em
  pano}%Titulo da Patente
{BR 10 2014 025133 2 A2}% Numero do pedido
{08/10/2014}% Data do Depósito
{03/05/2016}% Data da Publicação
{refere-se a um implemento agrícola acionado pela tomada de potência de um
  trator (1) capaz de enrolar os panos utilizados na operações de colheita por
  derriça de café, frutas e outros grãos, o pano contendo o produto colhido é
  fixado no mecanismo enrolador (4) que recebe movimento do trator (1) através
  do eixo cardan (5), da transmissão primária (6) e da transmissão secundária
  (7), ao ser enrolado o pano desliza sobre a rampa de recolhimento (3) e d
  posita os frutos e grãos colhidos em uma carreta (2) acoplada ao mesmo trator
  (1). Na forma usual os frutos e grãos derriçados são recolhidos manualmente, o
  que além de demandar muito tempo possui elevado custo.}% Resumo
{patente-inpi-01}% Figura da Patente

\PatenteINPI{Máquina derriçadora articulada tratorizada frutos de café e
  similares}% Titulo da Patente
{BR 10 2013 031650 4 A2}% Numero do pedido
{26/11/2013}% Data do Depósito
{10/11/2015}% Data da Publicação
{Consiste em uma máquina agrícola para derriça de frutos de café e similares
  como, azeitonas, blueberry, uvas, entre outras, em lavouras planas e de
  montanha com inclinações superiores a 20\%, com ruas ou entre linhas de
  plantio terraceadas, com espaçamento mínimo entre linhas de 3,00 metros,
  condições que são limitantes para os equipamentos atualmente fabricados. A
  derriçadora é dotada de hastes cilíndricas (48) dispostas ortogonalmente à
  placa metálica derriçadora (52), que pertence ao mecanismo de vibração (37),
  que é acionado pelo motor (43). O braço de extensão (6) será fixado no braço
  de levante (7), que será fixado no braço de giro (2) para garantir a
  movimentação nos eixos cartesianos "X,', "Y" e "Z", ocasionadas pela pressão
  positiva e negativa do fluido hidráulico nos cilindros hidráulicos de giro
  (12), de extensão (18) e de levante (3). O deslocamento do mecanismo de
  derriça (45) é realizado pelos braços (6, 7 e 2) e cilindros hidráulicos (12,
  18 e 3), para a lateral da linha de plantio em que se encontram os pés de
  café. O sistema de apoio (15) é composto pelo rodado (11), pela mola (13), as
  barras (16) e garfo (9). A Utilização desta derriçadora não esta restrita ás
  regiões montanhosas, mas sua inovação está em possibilitar a colheita
  mecanizada de frutos de café, em terrenos montanhosos com inclinações
  superiores aos 20\% em que a técnica do terraceamento possa ser executada,
  além de ser uma máquina de baixo custo, quando comparada às grandes
  colhedoras.}% Resumo
{patente-inpi-02}% Figura da patente

\PatenteINPI{Titulo}% Titulo da Patente
{Num Pedido}% Numero do pedido
{00/00/0000}% Data do Depósito
{00/00/0000}% Data da Publicação
{Resumo da patente...}% Resumo
{}% Figura da patente

%%% Local Variables:
%%% mode: latex
%%% TeX-master: "../template-01"
%%% End:
