
\chapter{ESPACENET Worldwide EN}
\label{chap:espacenet-en}

\section{IPC A01D 46/00}
\label{sec:ipc-a01d-4600}

\subsection{Harvesting AND Selective}
\label{sec:harvesting-selective}



\PatenteESPACENET{Selectively harvesting fruits}% Titulo da Patente
{WO2016171789 (A1)}% Numero da Patente
{2016-10-27}% Data da Patente
{A method of selectively harvesting fruits includes generating a first series of
  images of a plant while the plant is moved to expose hidden fruits,
  identifying an object displayed in the first series of images as a suspect
  fruit from feature boundary data defined by color regions, comparing a color
  parameter of the suspect fruit to a criterion associated with a ripeness,
  advancing an automated picker toward the suspect fruit based on a determined
  position of the suspect fruit and while monitoring a proximity sensor of the
  automated picker, detecting a color of the impediment and confirming that the
  impediment is a fruit, operating the automated picker to pick the fruit from
  the plant, and while the automated picker is operated to pick the fruit,
  generating a second series of images of one or more additional plants while
  the one or more additional plants are moved to expose additional
  fruits.}% Abstract
{patente-espacenet-en-01}% Figura


\PatenteESPACENET{Multi-robot crop harvesting machine}% Titulo da Patente
{WO2016132264 (A1)}% Numero da Patente
{2016-08-25}% Data da Patente
{A harvesting system (20) includes multiple robots (24), one or more sensors
  (48) and one or more computers (54, 58). The robots are mounted on a common
  frame (28) facing an area to be harvested, and are each configured to harvest
  crop items by reaching and gripping the crop items from a fixed angle of
  approach. The sensors are configured to acquire images of the area. The
  computers are configured to identify the crop items in the images, and to
  direct the robots to harvest the identified crop items.}% Abstract
{patente-espacenet-en-02}% Figura


\subsection{Harvesting AND Coffee}
\label{sec:harvesting-coffee-2}

\PatenteESPACENET{Air blast soft fruit harvesting device}% Titulo da Patente
{US2015223399 (A1)}% Numero da Patente
{2015-08-13}% Data da Patente
{The invention relates to the air blast soft fruit harvesting device comprising
  an air flow director (1), pulsator (2), distancing channel (3), segmented
  extension (4) and blower fan (5) assembly. The distancing channel (3) is
  connected with the segmented extension (4) on the top side and with the
  pulsator (2) on the bottom side, whereas the pulsator (2) is simultaneously
  connected with the air flow director (1) on the bottom side, and the segmented
  extension (4) is connected with the blower fan (5). Theblower fan (5)
  constantly sends air to the director (1), through the segmented extension (4),
  distancing channel (3) and pulsator (2) directing the air towards a fruit
  shrub, thus shaking ripe fruits.; In the course of the rotation of the
  pulsator (2) rotor (32), the pulsator (2) alternately sends air, in a single
  cycle, through the channels (6, 7 and 8), and then through the channels (9, 10
  and 11) of the director (1), where the air flow leaving the pulsator (2)
  varies, achieving its power impact. It is possible to regulate the air blast
  power and speed over the connecting tube (78) by discharging part of the air
  which the blower fan (5) sends towards the director (1), into the
  atmosphere. The regulation is performed manually with the lever (81).  Thus,
  regulation of the air blow power impact on a fruit shrub is obtained.  When
  using the device for harvesting soft fruit, two devices are used
  simultaneously, positioned in such way that there is a row of shrubs between
  them.}% Abstract
{patente-espacenet-en-03}% Figura


%\PatenteESPACENET{}% Titulo da Patente
% {}% Numero da Patente
% {}% Data da Patente
% {}% Abstract
% {}% Figura



%%% Local Variables:
%%% mode: latex
%%% TeX-master: "../template-01"
%%% End:
